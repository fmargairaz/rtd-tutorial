\chapter{QES-Plume}

\section{The model}

\section{The XML file}

\subsection{Simulation Parameters}

The parameters below are the parameters used to run the Plume model.

\begin{lstlisting}[language=XML]
<simulationParameters>
    <simDur> 1000.0 </simDur>   <!-- Total simulation time -->
    <timeStep> 0.1 </timeStep>  <!-- time step -->
    <CourantNumber> 0.5 </CourantNumber>
    <invarianceTol> 1e-10 </invarianceTol>
    <C_0> 4.0 </C_0>
    <updateFrequency_particleLoop> 10000 </updateFrequency_particleLoop>
    <updateFrequency_timeLoop> 100 </updateFrequency_timeLoop>
</simulationParameters>
\end{lstlisting}

\subsection{Collection Parameters}

The parameters below are the parameters used to calculate the concentration of particle (in \#particles/m$^{3}$). All parameters are in SI units by default (second and meters). The size of collection area should be set smaller or equal to the domain.
\begin{lstlisting}[language=XML]
<collectionParameters>
    <timeAvgStart> 0.0 </timeAvgStart>  <!-- time to start calc of concentration -->
    <timeAvgFreq> 60.0 </timeAvgFreq>   <!-- averaging period -->
    <boxBoundsX1> 0.0 </boxBoundsX1>    <!-- beginning of collection area -->
    <boxBoundsX2> 200.0 </boxBoundsX2>  <!-- end of collection area -->
    <boxBoundsY1> 0.0 </boxBoundsY1>
    <boxBoundsY2> 200.0 </boxBoundsY2>
    <boxBoundsZ1> 0.0 </boxBoundsZ1>
    <boxBoundsZ2> 200.0 </boxBoundsZ2>
    <nBoxesX> 200 </nBoxesX>            <!-- resolution of concentration -->
    <nBoxesY> 200 </nBoxesY>
    <nBoxesZ> 200 </nBoxesZ>
</collectionParameters>
\end{lstlisting}

\subsection{Sources}
\begin{lstlisting}[language=XML]
<sources>
    <numSources> 1 </numSources> <!-- Number of active sources -->
    <!-- HERE COME THE SOURCES -->
</sources>
\end{lstlisting}

\subsubsection{Source types}
\begin{lstlisting}[language=XML]
<SourcePoint>
    <!-- HERE COMES THE RELEASE TYPE -->
    <posX> 40.0 </posX>
    <posY> 80.0 </posY>
    <posZ> 30.0 </posZ>
</SourcePoint>
\end{lstlisting}

\begin{lstlisting}[language=XML]
<SourceLine>
    <!-- HERE COMES THE RELEASE TYPE -->
    <posX_0> 25.0 </posX_0>
    <posY_0> 175.0 </posY_0>
    <posZ_0> 40.0 </posZ_0>
    <posX_1> 50.0 </posX_1>
    <posY_1> 25.0 </posY_1>
    <posZ_1> 40.0 </posZ_1>
</SourceLine>
\end{lstlisting}

\begin{lstlisting}[language=XML]
<SourceCube>
    <!-- HERE COMES THE RELEASE TYPE -->
    <minX> 75.0 </minX>
    <minY> 25.0 </minY>
    <minZ> 70.0 </minZ>
    <maxX> 80.0 </maxX>
    <maxY> 35.0 </maxY>
    <maxZ> 80.0 </maxZ>
</SourceCube>
\end{lstlisting}

\begin{lstlisting}[language=XML]
<SourceCircle>
    <!-- HERE COMES THE RELEASE TYPE -->
    <posX> 40.0 </posX>
    <posY> 80.0 </posY>
    <posZ> 30.0 </posZ>
    <radius> 30.0 </radius>
</SourceCircle>
\end{lstlisting}

\begin{lstlisting}[language=XML]
<SourceFullDomain>
    <!-- HERE COMES THE RELEASE TYPE -->
</SourceFullDomain>
\end{lstlisting}

\subsubsection{Release types}
\begin{lstlisting}[language=XML]
<ReleaseType_continuous>
    <parPerTimestep>10</parPerTimestep>
</ReleaseType_continuous>
\end{lstlisting}

\begin{lstlisting}[language=XML]
<ReleaseType_duration>
    <releaseEndTime>0</releaseEndTime>
    <releaseEndTime>5</releaseEndTime>
    <parPerTimestep>10</parPerTimestep>
</ReleaseType_duration>
\end{lstlisting}

\begin{lstlisting}[language=XML]
<ReleaseType_instantaneous>
    <numPar>100000</numPar>
</ReleaseType_instantaneous>
\end{lstlisting}

\subsection{Boundary Conditions}
\begin{lstlisting}[language=XML]
<boundaryConditions>
    <xBCtype>exiting</xBCtype>
    <yBCtype>exiting</yBCtype>
    <zBCtype>exiting</zBCtype>
    <wallReflection>stairstepReflection</wallReflection>
</boundaryConditions>
\end{lstlisting}

\noindent Here are the option of the boundary conditions types:
\begin{itemize}
    \item \texttt{exiting} particle exit the domain
    \item \texttt{periodic} particle reenter the domain at the other side
    \item \texttt{reflection} particle is reflected from the domain boundary (works only of domain ends)
\end{itemize}

\noindent Here are the option of the wall reflections methods
\begin{itemize}
    \item \texttt{doNothing} nothing happen when particle enter wall
    \item \texttt{setInactive} (default) particle is set to inactive when entering a wall
    \item \texttt{stairstepReflection} particle use full stair step reflection when entering a wall
\end{itemize}


\subsection{Full XML Example}
\begin{lstlisting}[language=XML]
<simulationParameters>
    <simDur> 1000.0 </simDur>
    <timeStep> 0.1 </timeStep>
    <CourantNumber> 1 </CourantNumber>
    <invarianceTol> 1e-10 </invarianceTol>
    <C_0> 4.0 </C_0>
    <updateFrequency_particleLoop> 10000 </updateFrequency_particleLoop>
    <updateFrequency_timeLoop> 100 </updateFrequency_timeLoop>
</simulationParameters>
<collectionParameters>
    <timeAvgStart> 0.0 </timeAvgStart>
    <timeAvgFreq> 60.0 </timeAvgFreq>
    <boxBoundsX1> 0.0 </boxBoundsX1>
    <boxBoundsX2> 200.0 </boxBoundsX2>
    <boxBoundsY1> 0.0 </boxBoundsY1>
    <boxBoundsY2> 200.0 </boxBoundsY2>
    <boxBoundsZ1> 0.0 </boxBoundsZ1>
    <boxBoundsZ2> 200.0 </boxBoundsZ2>
    <nBoxesX> 200 </nBoxesX>
    <nBoxesY> 200 </nBoxesY>
    <nBoxesZ> 200 </nBoxesZ>
</collectionParameters>
<sources>
    <numSources> 1 </numSources>
    <SourcePoint>
        <ReleaseType_continuous>
            <parPerTimestep>10</parPerTimestep>
        </ReleaseType_continuous>
        <posX> 10.0 </posX>
        <posY> 100.0 </posY>
        <posZ> 50.0 </posZ>
  	</SourcePoint>
</sources>
<boundaryConditions>
    <xBCtype>exiting</xBCtype>
    <yBCtype>exiting</yBCtype>
    <zBCtype>exiting</zBCtype>
    <wallReflection>stairstepReflection</wallReflection>
</boundaryConditions>
\end{lstlisting}
